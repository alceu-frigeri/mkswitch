%%%==============================================================================
% WinEdt pragmas
% !Mode:: "TeX:EN"
% Default Compile engines:
% !TEX program = pdflatex
% !PDFTeXify ext =  --enable-etex  --restrict-write18
% !PDFLaTeX ext  =  --enable-etex  --restrict-write18
% !BIB program = biber
%%%==============================================================================
%% Copyright 2025-present by Alceu Frigeri
%%
%% This work may be distributed and/or modified under the conditions of
%%
%% * The [LaTeX Project Public License](http://www.latex-project.org/lppl.txt),
%%   version 1.3c (or later), and/or
%% * The [GNU Affero General Public License](https://www.gnu.org/licenses/agpl-3.0.html),
%%   version 3 (or later)
%%
%% This work has the LPPL maintenance status *maintained*.
%%
%% The Current Maintainer of this work is Alceu Frigeri
%%
%% This is version {1.1} {2025/05/11}
%%
%% The list of files that compose this work can be found in the README.md file at
%% https://ctan.org/pkg/switch
%%
%%%==============================================================================
\documentclass[10pt]{article}
\RequirePackage[verbose,a4paper,marginparwidth=27.5mm,top=2.5cm,bottom=1.5cm,hmargin={40mm,20mm},marginparsep=2.5mm,columnsep=10mm,asymmetric]{geometry}
\usepackage{codedescribe}
\RequirePackage[inline]{enumitem}
\SetEnumitemKey{miditemsep}{parsep=0ex,itemsep=0.4ex}

%%\usepackage[american,siunitx,cuteinductors,smartlabels,arrowmos,EFvoltages,betterproportions]{circuitikz}
%%\usetikzlibrary{math}

\usepackage{switch}

\RequirePackage[backend=biber]{biblatex}
\addbibresource{switch.bib}

\RequirePackage[hidelinks,hypertexnames=false]{hyperref}


\begin{document}
\tstitle{
  author={Alceu Frigeri\footnote{\tsverb{https://github.com/alceu-frigeri/switch}}},
  date={\tsdate},
  title={The switch Package\break Version \PkgInfo{switch}{version}}
  }
  

\begin{typesetabstract}
 
This package offers two commands aimed at implementing a switch/case alike command.
\end{typesetabstract}

\tableofcontents

\section{Introduction}
There are many ways of implementing a switch case programming structure. Notably, one can use \tsobj{\str_case:nn} from \tsobj[pkg]{expl3}, or go over a loop using \tsobj{\pdfstrcmp}, or construct an if-then-else tower, etc.

This implements a solution, based on  \cite{stackexchage:switchcase}, which (besides being simple) has the advantage of being constant time: once the cases are set up, suffice a single (internal) if (\tsobj{\ifcsname}) to select the correct code to be executed. 



\begin{tsremark}
  The implementation creates a \tsobj{\csname} for each case, and it uses (at the end) the primitive \tsobj{\ifcsname} to select the correct case.
\end{tsremark}
\begin{tsremark}
  The coding is done using \tsobj[pkg]{expl3}, just for the sake of readability, in the package comments one can find an implementation using just \TeX\  primitives.
\end{tsremark}

\setnewcodekey {switch} 
  {
    keywd2={newswitch,switch_new:Nn,addcase,switch_addcase:Nnn},
    emph={TextCase,myCase},
    emph2={CaseAstring},
    codeprefix={},
    resultprefix={}
  }


\section{Commands}
Two set of commands are created, one to be used in a \tsobj[pkg]{expl3} code régime, and another set to be used in a user document.

\subsection{User Document ones}
\begin{codedescribe}{\newswitch}
\begin{codesyntax}%
\tsobj{\newswitch} \tsmeta{switch} \tsargs[marg]{default-code}
\end{codesyntax}
It will create a new switch \tsobj[marg]{switch}, which will expects a single argument. In case the argument doesn't corresponds to any defined case, \tsobj[marg]{default-code} will be used. The resulting \tsobj[marg]{switch} command is expandable, if \tsobj[marg]{default-code} and \tsobj[marg]{case-code} (added by \tsobj{\addcase}) also are. This is just an alias for \tsobj{\switch_new:Nn}
\end{codedescribe}
\begin{tsremark}
  \verb|#1| can be used in \tsobj[marg]{default-code}. An error is raised if \tsobj[marg]{switch} is already defined.
\end{tsremark}

\begin{codedescribe}{\addcase}
\begin{codesyntax}%
\tsobj{\addcase} \tsmeta{switch} \tsargs[marg]{case,case-code}
\end{codesyntax}
It will add a \tsobj[marg]{case} to a previously defined \tsmeta{switch} and associates \tsobj[marg]{case-code} with it. \tsobj[marg]{case} will be fully expanded at definition time. Once defined one can call \tsverb[verb]{\switch{case}}, which will put said \tsobj[marg]{case-code} in the input stream.   This is just an alias for \tsobj{\switch_addcase:Nnn}.
\end{codedescribe}

\subsubsection{Example}

First we create a switch, and associate a few (or more) cases. Note the possibility of using an auxiliary (fully expandable) macro/command when defining the cases.
\begin{codestore}[switch02]
\def\CaseAstring{case-A}
\newswitch \myCase  {I~ don't~ know:~ #1\par}
\addcase   \myCase  {\CaseAstring} {A~ was~ used\par}
\addcase   \myCase  {case-B}       {B~ was~ used\par}
\end{codestore}

\begin{codestore}[switch02]
\def\somemacro{case-A}
\def\someothermacro{case-X}

If B, then \myCase{case-B}
If A, then \myCase{case-A}
If X, then \myCase{case-X}

if somemacro: \myCase{\somemacro}
if someothermacro: \myCase{\someothermacro}
\end{codestore}

\tscode*[switch]{switch02}[1]
\tsexec{switch02}[1]

To use the \tsobj[marg]{switch}, one just has to call it with \tsobj[marg]{case} as an argument. Note the possibility of using an auxiliary macro/command (which has to be fully expandable) as a \tsobj[marg]{case}.

\tsdemo[switch]{switch02}[2]


\subsection{Expl3 ones}
\begin{codedescribe}{\switch_new:Nn}
\begin{codesyntax}%
\tsobj{\switch_new:Nn} \tsmeta{switch} \tsargs[marg]{default-code}
\end{codesyntax}
It will create a new switch \tsobj[marg]{switch}, which will expects a single, type n, argument. In case the argument doesn't corresponds to any defined case, \tsobj[marg]{default-code} will be used. The resulting \tsobj[marg]{switch} command is expandable, if \tsobj[marg]{default-code} and \tsobj[marg]{case-code} (added by \tsobj{\switch_addcase:Nnn}) also are. 
\end{codedescribe}
\begin{tsremark}
  \verb|#1| can be used in \tsobj[marg]{default-code}.  An error is raised if \tsobj[marg]{switch} is already defined.
\end{tsremark}

\begin{codedescribe}{\switch_addcase:Nnn}
\begin{codesyntax}%
\tsobj{\switch_addcase:Nnn} \tsmeta{switch} \tsargs[marg]{case,case-code}
\end{codesyntax}
It will add a \tsobj[marg]{case} to a previously defined \tsmeta{switch} and associates \tsobj[marg]{case-code} with it. \tsobj[marg]{case} will be fully expanded at definition time. Once defined one can call \tsverb[verb]{\switch{case}}, which will put said \tsobj[marg]{case-code} in the input stream. 
\end{codedescribe}

\begin{codedescribe}[code,new=2025-05-13,EXP]{\switch_if_exist:NTF,\switch_if_case_exist:NnTF}
\begin{codesyntax}%
\tsobj{\switch_if_exist:NTF} \tsmeta{switch} \tsargs[marg]{if-true,if-false} 
\tsobj{\switch_if_case_exist:NnTF} \tsmeta{switch} \tsargs[marg]{case,if-true,if-false}
\end{codesyntax}
Tests if the \tsobj[marg]{switch}, or \tsobj[marg]{case}, are defined or not. It doesn't test if they are really a \tsobj[marg]{switch}/\tsobj[marg]{case}.
\end{codedescribe}

\begin{codedescribe}[code,new=2025-05-13]{\switch_undefine:N,\switch_case_undefine:Nn}
\begin{codesyntax}%
\tsobj{\switch_undefine:N} \tsmeta{switch}
\tsobj{\switch_case_undefine:Nn} \tsmeta{switch} \tsargs[marg]{case}
\end{codesyntax}
Undefine the \tsobj[marg]{switch} and/or specific \tsobj[marg]{case}. Please note, when undefining a \tsobj[marg]{switch}, the \tsverb[verb]{\csname} associated with the cases aren't undefined (if needed, they have to be undefined one by one).
\end{codedescribe}

\subsubsection{Example}

First we create a switch, and associate a few (or more) cases. Note the possibility of using an auxiliary (fully expandable) macro/command when defining the cases.
\begin{codestore}[switch01]
\ExplSyntaxOn
\def\CaseAstring{case-A}
\switch_new:Nn      \TextCase  {I~ don't~ know:~ #1\par}
\switch_addcase:Nnn \TextCase  {\CaseAstring} {A~ was~ used\par}
\switch_addcase:Nnn \TextCase  {case-B} {B~ was~ used\par}
\ExplSyntaxOff
\end{codestore}

\begin{codestore}[switch01]
\def\somemacro{case-A}
\def\someothermacro{case-X}
If B, then \TextCase{case-B}
If A, then \TextCase{case-A}
If X, then \TextCase{case-X}
if somemacro: \TextCase{\somemacro}
if someothermacro: \TextCase{\someothermacro}

\end{codestore}

\tscode*[switch]{switch01}[1]
\tsexec{switch01}[1]

To use the \tsobj[marg]{switch}, one just has to call it with \tsobj[marg]{case} as an argument. Note the possibility of using an auxiliary macro/command (which has to be fully expandable) as a \tsobj[marg]{case}.


\tsdemo[switch]{switch01}[2]

\section{Advanced Use}
Since the resulting \tsobj[marg]{switch} is fully expandable (if the provided \tsobj[marg]{case-code}s also are), one can design the \tsobj[marg]{case-code}s to absorb one or more parameter/tokens.

\begin{tsremark}[{\color{red}Careful:}]
  make sure that all \tsobj[marg]{case-code}s absorb the same number of parameters, to avoid ``leftovers'' or tricky errors.
\end{tsremark}

For instance, note the use of \tsobj{\@gobble} to absorb an unused parameter, or how \tsobj{\cmdY} is defined (with two parameters) then used with a ``fixed one''. The resulting command, \tsobj{\TCase}, absorbs 2 tokens/parameters:

\begin{codestore}[switch03]
\NewDocumentCommand \cmdX{m}   {I got #1}
\NewDocumentCommand \cmdY{mm}  {Two: #1 and #2}
\NewDocumentCommand \Astring{} {case-A}

\makeatletter
\newswitch \TCase       {I~ don't~ know:~ #1 \@gobble}
\makeatother
\addcase \TCase  {\Astring} {\cmdY{A~ given}}
\addcase \TCase  {case-B} {B~ was~ used. \cmdX}
\end{codestore}

\begin{codestore}[switch03]
If B, then \TCase{case-B}{extra-B}\par
If A, then \TCase{case-A}{extra-A}\par
If X, then \TCase{case-X}{extra-X}\par
\end{codestore}

\tscode*[switch,emph={TCase},emph2={cmdX,cmdY,@gobble}]{switch03}[1]
\tsexec{switch03}[1]



\tsdemo[switch,emph={TCase},emph2={cmdX,cmdY,@gobble}]{switch03}[2]

Needless to say, the same applies under \tsobj[pkg]{expl3}.


\begin{codestore}[switch04]
\ExplSyntaxOn
\cs_new:Npn \__cmdX:n   #1   {I~ got~ #1}
\cs_new:Npn \__cmdY:nn  #1#2 {Two:~ #1~ and~ #2}
\tl_new:N   \l__case_tl 
\tl_set:Nn  \l__case_tl {case-A}

\switch_new:Nn      \TxCase  {I~ don't~ know:~ #1 \use_none:n}
\switch_addcase:Nnn \TxCase  {\l__case_tl} {\__cmdY:nn{A~ given}}
\switch_addcase:Nnn \TxCase  {case-B} {B~ was~ used.~ \__cmdX:n}
\ExplSyntaxOff
\end{codestore}

\begin{codestore}[switch04]
If B, then \TxCase{case-B}{extra-B}\par
If A, then \TxCase{case-A}{extra-A}\par
If X, then \TxCase{case-X}{extra-X}\par
\end{codestore}

\tscode*[switch,emph={TxCase},emph2={cmdX,cmdY,use_none:n}]{switch04}[1]
\tsexec{switch04}[1]

\tsdemo[switch,emph={TxCase},emph2={cmdX,cmdY,@gobble}]{switch04}[2]

\printbibliography



\end{document} 